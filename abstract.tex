\begin{center}
	\begin{singlespace}
       ABSTRACT\\
       \vspace{0.5cm}
       THE COST OF REPRESSION: BRIBERY, COMPETENCE, AND INFORMATIONAL AUTOCRATS\\
       \vspace{0.5cm}
       by\\
       \vspace{0.5cm}
       Kaan Aksoy\\
       \vspace{0.5cm}
       The University of Wisconsin-Milwaukee, 2023\\
       Under the Supervision of Associate Professor Ora John Reuter\\
       \end{singlespace}    
       \vspace{2cm}	
       \justifying
As the third wave of democratisation slows down and seems to be reversing, we nevertheless observe that heavy-handed repression, mass killing, and regime-perpetrated violence is decreasing. Autocracies and autocracy-bordering hybrid regimes increasingly rely on smarter, less repressive, targeted, yet still undemocratic means of maintaining an uneven political playing field.

I utilise the theory of informational autocracy to focus on these regimes, where autocracies faced with a more affluent "informed elite" will resort to less direct means of repression. I propose that cracking down on low-level (or "street-level") bribery is an important and noticeable aspect of competence, and therefore that these regimes will seek to decrease low-level bribery to maintain an image of competence. Using data from the Varieties of Democracy Project, I test two hypotheses relating to this theory. I find that political violence has a positive relationship with low-level bribery.

My research contributes to the literature bridging authoritarianism and corruption, as well as literature on corruption itself by unpacking the conceptual bundle of "corruption" into its constituent parts, focusing on the political implications of one aspect of corruption. Therefore, it enables future research to focus on policy implications on how to tackle corruption more precisely.
\end{center}